\documentclass{extarticle}
\sloppy

%%%%%%%%%%%%%%%%%%%%%%%%%%%%%%%%%%%%%%%%%%%%%%%%%%%%%%%%%%%%%%%%%%%%%%
% PACKAGES            																						  %
%%%%%%%%%%%%%%%%%%%%%%%%%%%%%%%%%%%%%%%%%%%%%%%%%%%%%%%%%%%%%%%%%%%%%
\usepackage[10pt]{extsizes}
\usepackage{amsfonts}
\usepackage{amsthm}
\usepackage{amssymb}
\usepackage[shortlabels]{enumitem}
\usepackage{microtype} 
\usepackage{amsmath}
\usepackage{mathtools}
\usepackage{commath}

%%%%%%%%%%%%%%%%%%%%%%%%%%%%%%%%%%%%%%%%%%%%%%%%%%%%%%%%%%%%%%%%%%%%%%
% PROBLEM ENVIRONMENT         																			           %
%%%%%%%%%%%%%%%%%%%%%%%%%%%%%%%%%%%%%%%%%%%%%%%%%%%%%%%%%%%%%%%%%%%%%
\usepackage{tcolorbox}
\tcbuselibrary{theorems, breakable, skins}
\newtcbtheorem{prob}% environment name
              {Problem}% Title text
  {enhanced, % tcolorbox styles
  attach boxed title to top left={xshift = 4mm, yshift=-2mm},
  colback=blue!5, colframe=black, colbacktitle=blue!3, coltitle=black,
  boxed title style={size=small,colframe=gray},
  fonttitle=\bfseries,
  separator sign none
  }%
  {} 
\newenvironment{problem}[1]{\begin{prob*}{#1}{}}{\end{prob*}}

%%%%%%%%%%%%%%%%%%%%%%%%%%%%%%%%%%%%%%%%%%%%%%%%%%%%%%%%%%%%%%%%%%%%%%
% THEOREMS/LEMMAS/ETC.         																			  %
%%%%%%%%%%%%%%%%%%%%%%%%%%%%%%%%%%%%%%%%%%%%%%%%%%%%%%%%%%%%%%%%%%%%%%
\newtheorem{thm}{Theorem}
\newtheorem*{thm-non}{Theorem}
\newtheorem{lemma}[thm]{Lemma}
\newtheorem{corollary}[thm]{Corollary}
\newtheorem*{definition}{Definition}

%%%%%%%%%%%%%%%%%%%%%%%%%%%%%%%%%%%%%%%%%%%%%%%%%%%%%%%%%%%%%%%%%%%%%%
% MY COMMANDS   																						  %
%%%%%%%%%%%%%%%%%%%%%%%%%%%%%%%%%%%%%%%%%%%%%%%%%%%%%%%%%%%%%%%%%%%%%
\newcommand{\Z}{\mathbb{Z}}
\newcommand{\R}{\mathbb{R}}
\newcommand{\C}{\mathbb{C}}
\newcommand{\F}{\mathbb{F}}
\newcommand{\bigO}{\mathcal{O}}
\newcommand{\Real}{\mathcal{Re}}
\newcommand{\poly}{\mathcal{P}}
\newcommand{\mat}{\mathcal{M}}
\DeclareMathOperator{\Span}{span}
\newcommand{\Hom}{\mathcal{L}}
\DeclareMathOperator{\Null}{null}
\DeclareMathOperator{\Range}{range}
\newcommand{\defeq}{\vcentcolon=}
\newcommand\widebar[1]{\mathop{\overline{#1}}}
\newcommand{\restr}[1]{|_{#1}}
\DeclarePairedDelimiterX{\inp}[2]{\langle}{\rangle}{#1, #2}
\DeclarePairedDelimiter\Mod{\lvert}{\rvert}
\DeclarePairedDelimiter\Norm{\lVert}{\rVert}


%%%%%%%%%%%%%%%%%%%%%%%%%%%%%%%%%%%%%%%%%%%%%%%%%%%%%%%%%%%%%%%%%%%%%%
% SECTION NUMBERING																				           %
%%%%%%%%%%%%%%%%%%%%%%%%%%%%%%%%%%%%%%%%%%%%%%%%%%%%%%%%%%%%%%%%%%%%%%
\renewcommand\thesection{\Alph{section}:}



%%%%%%%%%%%%%%%%%%%%%%%%%%%%%%%%%%%%%%%%%%%%%%%%%%%%%%%%%%%%%%%%%%%%%%
% DOCUMENT START              																			           %
%%%%%%%%%%%%%%%%%%%%%%%%%%%%%%%%%%%%%%%%%%%%%%%%%%%%%%%%%%%%%%%%%%%%%%
\title{\vspace{-2em}Chapter 7: Operators on Inner Product Spaces}
\author{\emph{Linear Algebra Done Right}, by Sheldon Axler}
\date{}

\begin{document}
\maketitle



%%%%%%%%%%%%%%%%%%%%%%%%%%%%%%%%%%%%%%%%%%%%%%%%%%%%%%%%%%%%%%%%%%%%%
% SECTION A            																			           
%%%%%%%%%%%%%%%%%%%%%%%%%%%%%%%%%%%%%%%%%%%%%%%%%%%%%%%%%%%%%%%%%%%%%
\section{Self-Adjoint and Normal Operators}

% Problem 1
\begin{problem}{1}
Suppose $n$ is a positive integer.  Define $T\in\Hom(\F^n)$ by
\begin{align*}
T(z_1, \dots, z_n) = (0, z_1, \dots, z_{n-1}).
\end{align*}
Find a formula for $T^\ast(z_1,\dots, z_n)$.
\end{problem}
\begin{proof}
Fix $(y_1,\dots, y_n)\in\F^n$.  Then for all $(z_1,\dots, z_n)\in \F^n$, we have
\begin{align*}
\inp{(z_1,\dots, z_n)}{T^\ast(y_1,\dots, y_n)} &= \inp{T(z_1,\dots, z_n)}{(y_1,\dots, y_n)}\\
&= \inp{(0, z_1,\dots, z_{n-1})}{(y_1,\dots, y_n)}\\
&= z_1y_2 + z_2y_3 + \dots + z_{n-1}y_n\\
&=\inp{(z_1,\dots, z_{n-1}, z_n)}{(y_2, \dots, y_n, 0)}.
\end{align*}
Thus $T^*$ is the left-shift operator.  That is, for all $(z_1,\dots, z_n)\in \F^n$, we have
\begin{align*}
T^\ast(z_1,\dots, z_n) = (z_2, \dots, z_n, 0),
\end{align*}
as desired.
\end{proof}

% Problem 2
\begin{problem}{2}
Suppose $T\in\Hom(V)$ and $\lambda\in\F$.  Prove that $\lambda$ is an eigenvalue of $T$ if and only if $\bar{\lambda}$ is an eigenvalue of $T^\ast$.
\end{problem}
\begin{proof}
Suppose $\lambda$ is an eigenvalue of $T$.  Then there exists $v\in V$ such that $Tv = \lambda v$.  It follows
\begin{align*}
\lambda \text{ is not an eigenvalue of }T &\iff T - \lambda I\text{ is invertible}\\
&\iff S(T - \lambda I) = (T- \lambda I)S = I \\
&~~~~~~~~~~\text{ for some }S\in\Hom(V)\\
&\iff S^\ast(T^\ast - \lambda I)^\ast = (T- \lambda I)^\ast S^\ast = I^\ast\\
&~~~~~~~~~~\text{ for some }S^\ast\in\Hom(V)\\
&\iff (T - \lambda I)^\ast\text{ is invertible}\\
&\iff T^\ast - \bar{\lambda}I \text{ is invertible}\\
&\iff \bar{\lambda} \text{ is not an eigenvalue of }T^\ast.
\end{align*}
Since the first statement and the last statement are equivalent, so too are their contrapositives.  Hence $\lambda$ is an eigenvalue of $T$ if and only if $\bar{\lambda}$ is an eigenvalue of $T^\ast$, as was to be shown.
\end{proof}

% Problem 3
\begin{problem}{3}
Suppose $T\in\Hom(V)$ and $U$ is a subspace of $V$.  Prove that $U$ is invariant under $T$ if and only if $U^\perp$ is invariant under $T^\ast$.
\end{problem}
\begin{proof}
$(\Rightarrow)$ First suppose $U$ is invariant under $T$, and let $x\in U^\perp$.  For any $u\in U$, it follows
\begin{align*}
\inp{T^\ast x}{u} &= \inp{x}{Tu}\\
&= 0,
\end{align*}
where the second equality follows since $Tu\in U$ (by hypothesis).  Thus $T^\ast x\in U^\perp$ for all $x\in U^\perp$.  That is, $U^\perp$ is invariant under $T^\ast$.\\
\indent $(\Leftarrow)$ Now suppose $U^\perp$ is invariant under $T^\ast$, and let $y\in U$.  For any $u'\in U^\perp$, it follows
\begin{align*}
\inp{T y}{u'} &= \inp{y}{T^\ast u'}\\
&= 0,
\end{align*}
where the second equality follows since $T^\ast u'\in U^\perp$ (by hypothesis).  Thus $T y\in U$ for all $y\in U$.  That is, $U$ is invariant under $T$, completing the proof.
\end{proof}

\end{document}