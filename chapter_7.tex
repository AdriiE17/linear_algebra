\documentclass{extarticle}
\sloppy

%%%%%%%%%%%%%%%%%%%%%%%%%%%%%%%%%%%%%%%%%%%%%%%%%%%%%%%%%%%%%%%%%%%%%%
% PACKAGES            																						  %
%%%%%%%%%%%%%%%%%%%%%%%%%%%%%%%%%%%%%%%%%%%%%%%%%%%%%%%%%%%%%%%%%%%%%
\usepackage[10pt]{extsizes}
\usepackage{amsfonts}
\usepackage{amsthm}
\usepackage{amssymb}
\usepackage[shortlabels]{enumitem}
\usepackage{microtype} 
\usepackage{amsmath}
\usepackage{mathtools}
\usepackage{commath}

%%%%%%%%%%%%%%%%%%%%%%%%%%%%%%%%%%%%%%%%%%%%%%%%%%%%%%%%%%%%%%%%%%%%%%
% PROBLEM ENVIRONMENT         																			           %
%%%%%%%%%%%%%%%%%%%%%%%%%%%%%%%%%%%%%%%%%%%%%%%%%%%%%%%%%%%%%%%%%%%%%
\usepackage{tcolorbox}
\tcbuselibrary{theorems, breakable, skins}
\newtcbtheorem{prob}% environment name
              {Problem}% Title text
  {enhanced, % tcolorbox styles
  attach boxed title to top left={xshift = 4mm, yshift=-2mm},
  colback=blue!5, colframe=black, colbacktitle=blue!3, coltitle=black,
  boxed title style={size=small,colframe=gray},
  fonttitle=\bfseries,
  separator sign none
  }%
  {} 
\newenvironment{problem}[1]{\begin{prob*}{#1}{}}{\end{prob*}} 

%%%%%%%%%%%%%%%%%%%%%%%%%%%%%%%%%%%%%%%%%%%%%%%%%%%%%%%%%%%%%%%%%%%%%%
% THEOREMS/LEMMAS/ETC.         																			  %
%%%%%%%%%%%%%%%%%%%%%%%%%%%%%%%%%%%%%%%%%%%%%%%%%%%%%%%%%%%%%%%%%%%%%%
\newtheorem{thm}{Theorem}
\newtheorem*{thm-non}{Theorem}
\newtheorem{lemma}[thm]{Lemma}
\newtheorem{corollary}[thm]{Corollary}
\newtheorem*{definition}{Definition}

%%%%%%%%%%%%%%%%%%%%%%%%%%%%%%%%%%%%%%%%%%%%%%%%%%%%%%%%%%%%%%%%%%%%%%
% MY COMMANDS   																						  %
%%%%%%%%%%%%%%%%%%%%%%%%%%%%%%%%%%%%%%%%%%%%%%%%%%%%%%%%%%%%%%%%%%%%%
\newcommand{\Z}{\mathbb{Z}}
\newcommand{\R}{\mathbb{R}}
\newcommand{\C}{\mathbb{C}}
\newcommand{\F}{\mathbb{F}}
\newcommand{\bigO}{\mathcal{O}}
\newcommand{\Real}{\mathcal{Re}}
\newcommand{\poly}{\mathcal{P}}
\newcommand{\mat}{\mathcal{M}}
\DeclareMathOperator{\Span}{span}
\newcommand{\Hom}{\mathcal{L}}
\DeclareMathOperator{\Null}{null}
\DeclareMathOperator{\Range}{range}
\newcommand{\defeq}{\vcentcolon=}
\newcommand\widebar[1]{\mathop{\overline{#1}}}
\newcommand{\restr}[1]{|_{#1}}
\DeclarePairedDelimiterX{\inp}[2]{\langle}{\rangle}{#1, #2}
\DeclarePairedDelimiter\Mod{\lvert}{\rvert}
\DeclarePairedDelimiter\Norm{\lVert}{\rVert}


%%%%%%%%%%%%%%%%%%%%%%%%%%%%%%%%%%%%%%%%%%%%%%%%%%%%%%%%%%%%%%%%%%%%%%
% SECTION NUMBERING																				           %
%%%%%%%%%%%%%%%%%%%%%%%%%%%%%%%%%%%%%%%%%%%%%%%%%%%%%%%%%%%%%%%%%%%%%%
\renewcommand\thesection{\Alph{section}:}



%%%%%%%%%%%%%%%%%%%%%%%%%%%%%%%%%%%%%%%%%%%%%%%%%%%%%%%%%%%%%%%%%%%%%%
% DOCUMENT START              																			           %
%%%%%%%%%%%%%%%%%%%%%%%%%%%%%%%%%%%%%%%%%%%%%%%%%%%%%%%%%%%%%%%%%%%%%%
\title{\vspace{-2em}Chapter 7: Operators on Inner Product Spaces}
\author{\emph{Linear Algebra Done Right}, by Sheldon Axler}
\date{}

\begin{document}
\maketitle



%%%%%%%%%%%%%%%%%%%%%%%%%%%%%%%%%%%%%%%%%%%%%%%%%%%%%%%%%%%%%%%%%%%%%
% SECTION A            																			           
%%%%%%%%%%%%%%%%%%%%%%%%%%%%%%%%%%%%%%%%%%%%%%%%%%%%%%%%%%%%%%%%%%%%%
\section{Self-Adjoint and Normal Operators}

% Problem 1
\begin{problem}{1}
Suppose $n$ is a positive integer.  Define $T\in\Hom(\F^n)$ by
\begin{align*}
T(z_1, \dots, z_n) = (0, z_1, \dots, z_{n-1}).
\end{align*}
Find a formula for $T^\ast(z_1,\dots, z_n)$.
\end{problem}
\begin{proof}
Fix $(y_1,\dots, y_n)\in\F^n$.  Then for all $(z_1,\dots, z_n)\in \F^n$, we have
\begin{align*}
\inp{(z_1,\dots, z_n)}{T^\ast(y_1,\dots, y_n)} &= \inp{T(z_1,\dots, z_n)}{(y_1,\dots, y_n)}\\
&= \inp{(0, z_1,\dots, z_{n-1})}{(y_1,\dots, y_n)}\\
&= z_1y_2 + z_2y_3 + \dots + z_{n-1}y_n\\
&=\inp{(z_1,\dots, z_{n-1}, z_n)}{(y_2, \dots, y_n, 0)}.
\end{align*}
Thus $T^*$ is the left-shift operator.  That is, for all $(z_1,\dots, z_n)\in \F^n$, we have
\begin{align*}
T^\ast(z_1,\dots, z_n) = (z_2, \dots, z_n, 0),
\end{align*}
as desired.
\end{proof}

% Problem 2
\begin{problem}{2}
Suppose $T\in\Hom(V)$ and $\lambda\in\F$.  Prove that $\lambda$ is an eigenvalue of $T$ if and only if $\bar{\lambda}$ is an eigenvalue of $T^\ast$.
\end{problem}
\begin{proof}
Suppose $\lambda$ is an eigenvalue of $T$.  Then there exists $v\in V$ such that $Tv = \lambda v$.  It follows
\begin{align*}
\lambda \text{ is not an eigenvalue of }T &\iff T - \lambda I\text{ is invertible}\\
&\iff S(T - \lambda I) = (T- \lambda I)S = I \\
&~~~~~~~~~~\text{ for some }S\in\Hom(V)\\
&\iff S^\ast(T^\ast - \lambda I)^\ast = (T- \lambda I)^\ast S^\ast = I^\ast\\
&~~~~~~~~~~\text{ for some }S^\ast\in\Hom(V)\\
&\iff (T - \lambda I)^\ast\text{ is invertible}\\
&\iff T^\ast - \bar{\lambda}I \text{ is invertible}\\
&\iff \bar{\lambda} \text{ is not an eigenvalue of }T^\ast.
\end{align*}
Since the first statement and the last statement are equivalent, so too are their contrapositives.  Hence $\lambda$ is an eigenvalue of $T$ if and only if $\bar{\lambda}$ is an eigenvalue of $T^\ast$, as was to be shown.
\end{proof}

% Problem 3
\begin{problem}{3}
Suppose $T\in\Hom(V)$ and $U$ is a subspace of $V$.  Prove that $U$ is invariant under $T$ if and only if $U^\perp$ is invariant under $T^\ast$.
\end{problem}
\begin{proof}
$(\Rightarrow)$ First suppose $U$ is invariant under $T$, and let $x\in U^\perp$.  For any $u\in U$, it follows
\begin{align*}
\inp{T^\ast x}{u} &= \inp{x}{Tu}\\
&= 0,
\end{align*}
where the second equality follows since $Tu\in U$ (by hypothesis).  Thus $T^\ast x\in U^\perp$ for all $x\in U^\perp$.  That is, $U^\perp$ is invariant under $T^\ast$.\\
\indent $(\Leftarrow)$ Now suppose $U^\perp$ is invariant under $T^\ast$, and let $y\in U$.  For any $u'\in U^\perp$, it follows
\begin{align*}
\inp{T y}{u'} &= \inp{y}{T^\ast u'}\\
&= 0,
\end{align*}
where the second equality follows since $T^\ast u'\in U^\perp$ (by hypothesis).  Thus $T y\in U$ for all $y\in U$.  That is, $U$ is invariant under $T$, completing the proof.
\end{proof}

% Problem 5
\begin{problem}{5}
Prove that 
\begin{align*}
\dim\Null T^\ast = \dim\Null T + \dim W - \dim V
\end{align*}
and
\begin{align*}
\dim\Range T^* = \dim\Range T
\end{align*}
for every $T\in\Hom(V, W)$.
\end{problem}
\begin{proof}
Let $T\in\Hom(V, W)$.  Notice
\begin{align*}
\dim\Null T^\ast &= \dim\left(\Range T\right)^\perp\\
&=\dim W - \dim\Range T\\
&=\dim W + \dim\Null T - \dim V,
\end{align*}
where the first equality follows by 7.7(a), the second equality follows by 6.50, and the third equality follows by the Fundamental Theorem of Linear Maps.  Next notice
\begin{align*}
\dim\Range T^\ast &= \dim\left(\Null T\right)^\perp\\
&= \dim V - \dim\Null T\\
&= \dim \Range T,
\end{align*}
where the first equality follows by 7.7(b), and the second and third equalities follow again by the same theorems above.
\end{proof}

% Problem 7
\begin{problem}{7}
Suppose $S,T\in\Hom(V)$ are self-adjoint.  Prove that $ST$ is self-adjoint if and only if $ST=TS$.
\end{problem}
\begin{proof}
$(\Rightarrow)$ Suppose $ST$ is self-adjoint.  We have
\begin{align*}
ST &= (ST)^\ast\\
&= T^\ast S^\ast\\
&=TS,
\end{align*}
where the second equality follows by 7.6(e).\\
\indent $(\Leftarrow)$ Conversely, suppose $ST = TS$.  It follows
\begin{align*}
(ST)^\ast &= (TS)^\ast\\
&= S^\ast T^\ast,
\end{align*}
where the second equality again follows by 7.6(e), completing the proof.
\end{proof}

% Problem 9
\begin{problem}{9}
Suppose $V$ is a complex inner product space with $V\neq\{0\}$.  Show that the set of self-adjoint operators on $V$ is not a subspace of $\Hom(V)$.
\end{problem}
\begin{proof}
Let $\mathcal{A}$ denote the set of self-adjoint operators on $V$, and suppose $T\in\mathcal{A}$.  By 7.6(b), notice $(iT)^\ast = -iT^\ast$, so that $\mathcal{A}$ is not closed under scalar multiplication.  Thus $\mathcal{A}$ is not a subspace of $\Hom(V)$.
\end{proof}

% Problem 11
\begin{problem}{11}
Suppose $P\in\Hom(V)$ is such that $P^2 = P$.  Prove that there is a subspace $U$ of $V$ such that $P = P_U$ if and only if $P$ is self-adjoint.
\end{problem}
\begin{proof}
$(\Rightarrow)$ First suppose there is a subspace $U\subseteq V$ such that $P = P_U$, and let $v_1,v_2\in V$.  It follows
\begin{align*}
\inp{Pv_1}{v_2} &= \inp{u_1}{u_2 + w_2}\\
&= \inp{u_1}{u_2} + \inp{u_1}{w_2}\\
&= \inp{u_1}{u_2}\\
&= \inp{u_1}{u_2} + \inp{w_1}{u_2}\\
&= \inp{u_1 + w_1}{u_2}\\
&= \inp{v_1}{Pv_2},
\end{align*}
and thus $P = P^\ast$.\\
\indent $(\Leftarrow)$ Conversely, suppose $P = P^\ast$.  Let $v\in V$.  Notice $P(v - Pv)= Pv - P^2v = 0$, and hence $v - Pv\in\Null P$.  By 7.7(c), we know $\Null P = \left(\Range T^\ast\right)^\perp$.  By hypothesis, $P$ is self-adjoint, and hence we have $v-Pv \in \left(\Range T\right)^\perp$.  Notice we may write
\begin{align*}
v = Pv + (v - Pv),
\end{align*}
where $Pv \in \Range P$ and $v-Pv \in \left(\Range T\right)^\perp$.  Let $U = \Range P$.  Since the above holds for all $v\in V$, we conclude $P = P_U$, and the proof is complete.
\end{proof}

% Problem 13
\begin{problem}{13}
Give an example of an operator $T\in\Hom(\C^4)$ such that $T$ is normal but not self-adjoint.
\end{problem}
\begin{proof}
Let $T$ be the operator on $\C^4$ whose matrix with respect to the standard basis is
\begin{align*}
\begin{bmatrix}
2 & -3 & 0 & 0\\
3 &  2 & 0 & 0\\
0 &  0 & 0 & 0\\
0 &  0 & 0 & 0
\end{bmatrix}.
\end{align*} 
We claim $T$ is normal and not self-adjoint.  To see that $T$ is not self-adjoint, notice that the entry in row $2$, column $1$ does not equal the complex conjugate of the entry in row $1$ column $2$.\\
\indent Next, notice
\begin{align*}
\mat(TT^\ast) &= 
\begin{bmatrix}
2 & -3 & 0 & 0\\
3 &  2 & 0 & 0\\
0 &  0 & 0 & 0\\
0 &  0 & 0 & 0
\end{bmatrix}
\begin{bmatrix}
2  & 3 & 0 & 0\\
-3 & 2 & 0 & 0\\
0  & 0 & 0 & 0\\
0  & 0 & 0 & 0
\end{bmatrix}
= 
\begin{bmatrix}
13  & 0 & 0 & 0\\
0 & 13 & 0 & 0\\
0  & 0 & 0 & 0\\
0  & 0 & 0 & 0
\end{bmatrix}
\end{align*}
and 
\begin{align*}
\mat(T^\ast T) &= 
\begin{bmatrix}
2 & 3 & 0 & 0\\
-3 &  2 & 0 & 0\\
0 &  0 & 0 & 0\\
0 &  0 & 0 & 0
\end{bmatrix}
\begin{bmatrix}
2  & -3 & 0 & 0\\
3 & 2 & 0 & 0\\
0  & 0 & 0 & 0\\
0  & 0 & 0 & 0
\end{bmatrix}
= 
\begin{bmatrix}
13  & 0 & 0 & 0\\
0 & 13 & 0 & 0\\
0  & 0 & 0 & 0\\
0  & 0 & 0 & 0
\end{bmatrix},
\end{align*}
and hence $TT^\ast$ and $T^\ast T$ have the same matrix.  Thus $TT^\ast = T^\ast T$, and $T$ is normal.   
\end{proof}

% Problem 15
\begin{problem}{15}
Fix $u,x\in V$.  Define $T\in\Hom(V)$ by
\begin{align*}
Tv = \inp{v}{u}x
\end{align*}
for every $v\in V$.
\begin{itemize}[(a)]
\item Suppose $\F = \R$.  Prove that $T$ is self-adjoint if and only if $u,x$ is linearly dependent.
\item Prove that $T$ is normal if and only if $u,x$ is linearly dependent.
\end{itemize}
\end{problem}
\begin{proof}
We first derive a useful formula for $T^\ast$ which we'll use in both (a) and (b).  Let $w_1,w_2\in V$ and notice
\begin{align*}
\inp{w_1}{T^\ast w_2} &= \inp{Tw_1}{w_2}\\
&= \inp{\inp{w_1}{u}x}{w_2}\\
&= \inp{w_1}{u}\inp{x}{w_2}\\
&= \inp{w_1}{\widebar{\inp{x}{w_2}}u}\\
&= \inp{w_1}{\inp{w_2}{x}u},
\end{align*}
and thus $T^\ast w_2 = \inp{w_2}{x}u$.  Since $w_2$ was arbitrary, we may rewrite this as $T^\ast v = \inp{v}{x}u$ for all $v\in V$.
\begin{enumerate}[(a), listparindent=1.5em]
\item $(\Rightarrow)$ Suppose $T$ is self-adjoint.  Then we have
\begin{align*}
\inp{v}{u}x - \inp{v}{x}u = Tv - T^\ast v = 0
\end{align*}
for all $v\in V$.  In particular, we have
\begin{align*}
\inp{u}{u}x - \inp{u}{x}u = 0.
\end{align*}
We may assume both $u$ and $x$ are nonzero, for otherwise there is nothing to prove.  Hence $\inp{u}{u}\neq 0$, which forces $\inp{u}{x}$ to be nonzero as well, and thus the equation above shows $u,x$ is linearly dependent.\\
\indent $(\Leftarrow)$ Conversely, suppose $u,x$ is linearly dependent.  We may again assume both $u$ and $x$ are nonzero, for otherwise $T=0$, which is self-adjoint.  Thus there exists a nonzero $\alpha\in\R$ such that $u = \alpha x$.  It follows
\begin{align*}
Tv &= \inp{v}{u}x\\
&= \inp{v}{\alpha x}\frac{1}{\alpha} u\\
&= \inp{v}{x}u\\
&= T^\ast,
\end{align*}
and thus $T$ is self-adjoint, completing the proof.
\item $(\Rightarrow)$ Suppose $T$ is normal and let $v\in V$.  It follows
\begin{align*}
\inp{\inp{v}{u}x}{x}u &= T^\ast(\inp{v}{u}x)\\
&= T^\ast Tv\\
&= TT^\ast v\\
&= T(\inp{v}{x}u)\\
&= \inp{\inp{v}{x}u}{u}x.
\end{align*}
We may assume both $u$ and $x$ are nonzero, for otherwise there is nothing to prove.  Since the above holds for $v = u$, we may conclude $\inp{\inp{v}{u}x}{x} \neq 0$, which also forces $\inp{\inp{v}{x}u}{u} \neq 0$.  Thus $u, x$ is linearly dependent.\\
\indent $(\Leftarrow)$ Conversely, suppose $u,x$ is linearly dependent.  We may again assume both $u$ and $x$ are nonzero, for otherwise $T = 0$, which is normal.  Thus there exists a nonzero $\alpha\in\R$ such that $u = \alpha x$.  It follows
\begin{align*}
T^\ast Tv &= T^\ast(\inp{v}{u}x)\\
&= \inp{\inp{v}{u}x}{x}u\\
&= \left\langle\inp{v}{\alpha x}\frac{1}{\alpha} u, \frac{1}{\alpha}u\right\rangle \alpha x\\
&= \inp{\inp{v}{x}u}{u}x\\
&= T(\inp{v}{x}u)\\
&= TT^\ast v,
\end{align*}
and thus $T$ is normal, completing the proof. \qedhere
\end{enumerate}
\end{proof}

% Problem 16
\begin{problem}{16}
Suppose $T\in\Hom(V)$ is normal.  Prove that
\begin{align*}
\Range T = \Range T^\ast.
\end{align*}
\end{problem}
\begin{proof}
Suppose $T\in\Hom(V)$ is normal.  We first prove $\Null T = \Null T^\ast$.  It follows
\begin{align*}
v\in\Null T &\iff Tv = 0\\
&\iff \norm{Tv} = 0\\
&\iff \norm{T^\ast v} = 0\\
&\iff T^\ast v = 0\\
&\iff v\in\Null T^\ast,
\end{align*}
where the third equivalence follows by 7.20, and indeed we have $\Null T = \Null T^\ast$.  This implies $(\Null T)^\perp = (\Null T^\ast)^\perp$,
and by 7.7(b) and 7.7(c), this is equivalent to $\Range T^\ast = \Range T$, as desired.
\end{proof}

% Problem 17
\begin{problem}{17}
Suppose $T\in\Hom(V)$ is normal.  Prove that
\begin{align*}
\Null T^k = \Null T \quad \text{and} \quad \Range T^k = \Range T
\end{align*}
for every positive integer $k$.
\end{problem}
\begin{proof}
To show $\Null T^k = \Null T$, we first prove $\Null T^k = \Null T^{k + 1}$ for all $k\in\Z^+$.  Let $m\in\Z^+$.  If $m = 1$, there's nothing to prove, so we may assume $m > 1$.  Clearly, if $v\in\Null T^m$, then $v\in\Null T^{m+1}$, and hence $\Null T^m \subseteq\Null T^{m+1}$.  Next, suppose $v\in\Null T^{m + 1}$.  Then $T(T^m v)=0$, and hence $T^mv\in\Null T$.  By Problem 16, this implies $T^mv\in\Null T^\ast$, and by 7.7(a) we have $T^m\in(\Range T)^\perp$.  Since of course $T^mv\in\Range T$ as well, we must have $T^mv = 0$.  Thus $v \in \Null T^m$, and therefore $\Null T^{m+1} \subseteq \Null T^m$.  Thus $\Null T^m = \Null T^{m + 1}$.  Since $m$ was arbitrary, this implies $\Null T^k = \Null T$ for all $k\in\Z^+$, as desired.\\
\indent Now we will show $\Range T^k = \Range T$ for all $k\in\Z^+$.  Let $n\in\Z^+$.  If $n=1$, there's nothing to prove, so we may assume $n>1$.  Suppose $w\in \Range T^{n}$.  Then there exists $v\in V$ such that $T^nv = w$, and hence $T(T^{n-1}v) = w$, so that $w\in\Range T$ as well and we have $\Range T^n\subseteq \Range T$.  Next, notice 
\begin{align*}
\dim \Range T^n &= \dim V - \dim \Null T^n\\
&= \dim V - \dim\Null T\\
&= \dim \Range T,
\end{align*}
where the second equality follows from the previous paragraph.  Since $\Range T^n$ is a subspace of $\Range T$ of the same dimension, it must equal $\Range T$.  And since $n$ was arbitrary, we conclude $\Range T^k = \Range T$ for all $k\in\Z^+$, completing the proof.
\end{proof}

% Problem 19
\begin{problem}{19}
Suppose $T\in\Hom(\C^3)$ is normal and $T(1, 1, 1) = (2, 2, 2)$.  Suppose $(z_1,z_2,z_3)\in\Null T$.  Prove that $z_1 + z_2 + z_3 = 0$.
\end{problem}
\begin{proof}
By Problem 16, $\Null T = \Null T^\ast$, hence $T^\ast(z_1, z_2, z_3) = 0$.  Therefore, we have
\begin{align*}
2(z_1 + z_2 + z_3) &= \inp{(2, 2, 2)}{(z_1, z_2, z_3)}\\
&= \inp{T(1, 1, 1)}{(z_1, z_2, z_3)} \\
&= \inp{(1, 1, 1)}{T^\ast(z_1, z_2, z_3)}\\
&= \inp{(1, 1, 1)}{(0, 0, 0)}\\
&= 0,
\end{align*}
and so $z_1 + z_2 + z_3 = 0$, as was to be shown. 
\end{proof}

% Problem 21
\begin{problem}{21}
Fix a positive integer $n$.  In the inner product space of continuous real-valued functions on $[-\pi, \pi]$ with inner product
\begin{align*}
\inp{f}{g} = \int_{-\pi}^\pi f(x)g(x)dx,
\end{align*}
let 
\begin{align*}
V = \Span(1, \cos x, \cos 2x, \dots, \cos nx, \sin x, \sin 2x, \dots, \sin nx).
\end{align*}
\begin{enumerate}[(a)]
\item Define $D\in\Hom(V)$ by $Df = f'$.  Show that $D^\ast = -D$.  Conclude that $D$ is normal but not self-adjoint.
\item Define $T\in\Hom(V)$ by $Tf = f''$.  Show that $T$ is self-adjoint.
\end{enumerate}
\end{problem}
\begin{proof}
From Problem 4 of 6B, recall that
\begin{align*}
\frac{1}{\sqrt{2\pi}}, \frac{\cos x}{\sqrt{\pi}}, \frac{\cos 2x}{\sqrt{\pi}}, \dots, \frac{\cos nx}{\sqrt{\pi}}, \frac{\sin x}{\sqrt{\pi}}, \frac{\sin 2x}{\sqrt{\pi}}, \dots, \frac{\sin nx}{\sqrt{\pi}}
\end{align*}
is an orthonormal list, and hence it is an orthonormal basis of $V$.  
\begin{enumerate}[(a)]
\item For $k = 1,\dots, n$, define
\begin{align*}
e_k = \frac{\cos(kx)}{\sqrt{\pi}}\quad\text{and}\quad f_k= \frac{\sin(kx)}{\sqrt{\pi}}.
\end{align*}
Notice
\begin{align*}
De_k = -\frac{k\sin(kx)}{\sqrt{\pi}} = -kf_k\quad\text{and}\quad Df_k = \frac{k\cos(kx)}{\sqrt{\pi}} = ke_k,
\end{align*}
and thus, for any $v,w\in V$, it follows
\begin{align*}
\inp{v}{D^\ast w} &= \inp{Dv}{w}\\
&= \left\langle D\left(\left\langle v, \frac{1}{\sqrt{2\pi}}\right\rangle \frac{1}{\sqrt{2\pi}}+\sum_{k=1}^n\left(\inp{v}{e_k}e_k + \inp{v}{f_k}f_k\right)\right), w\right\rangle\\
&= \left\langle \sum_{k=1}^n\left(-k\inp{v}{e_k}f_k + k\inp{v}{f_k}e_k\right), w\right\rangle\\
&= -\sum_{k = 1}^nk\inp{v}{e_k}\inp{f_k}{w} + \sum_{k = 1}^nk\inp{v}{f_k}\inp{e_k}{w}\\
&= -\sum_{k= 1}^nk\inp{w}{f_k}\inp{v}{e_k} +  \sum_{k = 1}^nk\inp{w}{e_k}\inp{v}{f_k}\\
&= \sum_{k = 1}^nk\inp{w}{e_k}\inp{v}{f_k} - \sum_{k= 1}^nk\inp{w}{f_k}\inp{v}{e_k}\\
&= \left\langle v, \sum_{k=1}^nk\inp{w}{e_k}f_k \right\rangle - \left\langle v, \sum_{k=1}^nk\inp{w}{f_k}e_k \right\rangle\\
&= \left\langle v, \sum_{k=1}^n\left(k\inp{w}{e_k}f_k - k\inp{w}{f_k}e_k \right)\right\rangle\\
&= \left\langle v,  -D\left(\left\langle w, \frac{1}{\sqrt{2\pi}}\right\rangle \frac{1}{\sqrt{2\pi}}+\sum_{k=1}^n\left(\inp{w}{e_k}e_k + \inp{w}{f_k}f_k\right)\right)\right\rangle\\
&= \inp{v}{-Dw},
\end{align*}
and thus $D^\ast = -D$, showing that $D$ is not self-adjoint.  Moreover, notice that this implies
\begin{align*}
DD^\ast = D(-D) = -DD = (D^\ast)D = D^\ast D,
\end{align*}
so that $D$ is normal, completing the proof.

\item Notice $T = D^2$, and hence
\begin{align*} 
T^\ast = (DD)^\ast = D^\ast D^\ast = (-D)(-D) = D^2  = T.
\end{align*}
Thus $T$ is self-adjoint. \qedhere
\end{enumerate}
\end{proof}


%%%%%%%%%%%%%%%%%%%%%%%%%%%%%%%%%%%%%%%%%%%%%%%%%%%%%%%%%%%%%%%%%%%%%
% SECTION B           																			           
%%%%%%%%%%%%%%%%%%%%%%%%%%%%%%%%%%%%%%%%%%%%%%%%%%%%%%%%%%%%%%%%%%%%%
\section{The Spectral Theorem}

% Problem 1
\begin{problem}{1}
True or false (and give a proof of your answer): There exists $T\in\Hom(\R^3)$ such that $T$ is not self-adjoint (with respect to the usual inner product) and such that there is a basis of $\R^3$ consisting of eigenvectors of $T$.
\end{problem}
\begin{proof}
The statement is true.  To see this, consider the linear operator $T$ defined by its action on the basis $(1, 0, 0), (0, 1, 0), (0, 1, 1)$:
\begin{align*}
T(1, 0, 0) &= (0, 0, 0)\\
T(0, 1, 0) &= (0, 0, 0)\\
T(0, 1, 1) &= (0, 1, 1).
\end{align*}
Notice $T(1, 0, 0) = 0\cdot (1, 0, 0)$ and $T(0, 1, 0) = 0\cdot (0, 1, 0)$, so that $(1, 0, 0)$ and $(0, 1, 0)$ are eigenvectors with eigenvalue $0$.  Also, $(0, 1, 1)$ is an eigenvector with eigenvalue $1$.  Thus $(1, 0, 0,), (0, 1, 0), (0, 1, 1)$ is a basis of $\R^3$ consisting of eigenvectors of $T$.  That $T$ is not self-adjoint follows from the contrapositive of 7.22, since $(0, 1, 0)$ and $(0, 1, 1)$ correspond to distinct eigenvalues yet they are not orthogonal.
\end{proof}

% Problem 3
\begin{problem}{3}
Give an example of an operator $T\in\Hom(\C^3)$ such that $2$ and $3$ are the only eigenvalues of $T$ and $T^2 - 5T + 6I \neq 0$.
\end{problem}
\begin{proof}
Define $T\in\Hom(\C^3)$ by its action on the standard basis:
\begin{align*}
Te_1 &= 2e_2\\
Te_2 &= e_1 + 2e_2\\
Te_3 &= 3e_3.
\end{align*}
Then 
\begin{align*}
\mat(T) = \begin{bmatrix}2 & 1 & 0\\ 0 & 2 & 0\\ 0 & 0 & 3\end{bmatrix}.
\end{align*}
By 5.32, the only eigenvalues of $T$ are the entries on the diagonal: $2$ and $3$.  Now notice
\begin{align*}
(T^2 - 5T + 6I)e_2 &= (T - 3I)(T - 2I)e_2\\
&= (T - 3I)(Te_2 - 2e_2)\\
&= (T - 3I)(e_1 + 2e_2 - 2e_2)\\
&= (T - 3I)e_1\\
&= Te_1 - 3e_1\\
&= -e_1,
\end{align*}
so that $T^2 - 5T + 6I\neq 0$.  Thus $T$ is an operator of the desired form.
\end{proof}

% Problem 5
\begin{problem}{5}
Suppose $\F=\R$ and $T\in\Hom(V)$.  Prove that $T$ is self-adjoint if and only if all pairs of eigenvectors corresponding to distinct eigenvalues of $T$ are orthogonal and 
\begin{align*}
V = E(\lambda_1, T)\oplus \dots \oplus E(\lambda_m, T),
\end{align*}
where $\lambda_1,\dots,\lambda_m$ denote the distinct eigenvalues of $T$.
\end{problem}
\begin{proof}
$(\Leftarrow)$ Suppose all pairs of eigenvectors corresponding to distinct eigenvalues of $T$ are orthogonal and 
\begin{align*}
V = E(\lambda_1, T)\oplus \dots \oplus E(\lambda_m, T),
\end{align*}
where $\lambda_1,\dots,\lambda_m$ denote the distinct eigenvalues of $T$.  By 5.41, $V$ has a basis consisting of eigenvectors of $T$.  Dividing each element of the basis by its norm produces an orthonormal basis consisting of eigenvectors of $T$.  By the Real Spectral Theorem, $T$ is self-adjoint, as desired.\\
\indent $(\Rightarrow)$ Conversely, suppose $T$ is self-adjoint as suppose $v_1,v_2\in V$ are eigenvectors of $T$ corresponding to eigenvalues $\lambda_1,\lambda_2\in\R$ such that $\lambda_1\neq\lambda_2$.  It follows
\begin{align*}
0 &= \inp{Tv_1}{v_2} - \inp{v_1}{Tv_2}\\
&= \inp{\lambda_1 v_1}{v_2} - \inp{v_1}{\lambda_2 v_2}\\
&= \lambda_1\inp{v_1}{v_2} - \widebar{\lambda_2}\inp{v_1}{v_2}\\
&= \lambda_1\inp{v_1}{v_2} - \lambda_2\inp{v_1}{v_2}\\
&= (\lambda_1 - \lambda_2)\inp{v_1}{v_2}.
\end{align*}
Since $\lambda_1\neq \lambda_2$, it must be that $\inp{v_1}{v_2}=0$.  Thus all pairs of eigenvectors corresponding to distinct eigenvalues of $T$ are orthogonal.  By the Real Spectral Theorem, since $T$ is self-adjoint, $T$ is diagonalizable.  And by 5.34, this implies
\begin{align*}
V = E(\lambda_1, T)\oplus \dots \oplus E(\lambda_m, T),
\end{align*}
where $\lambda_1,\dots,\lambda_m$ denote the distinct eigenvalues of $T$, completing the proof.
\end{proof}

% Problem 6
\begin{problem}{6}
Prove that a normal operator on a complex vector space is self-adjoint if and only if all its eigenvalues are real.
\end{problem}
\begin{proof}
Let $T$ be a normal operator on a complex vector space, $V$.\\
\indent $(\Rightarrow)$ Suppose $T$ is self-adjoint.  Then by 7.13, all eigenvalues of $T$ are real.\\
\indent $(\Leftarrow)$ Conversely, suppose all eigenvalues of $T$ are real.  By the Complex Spectral Theorem, there exists an orthonormal basis $v_1,\dots, v_n$ of $V$ consisting of eigenvectors of $T$.  Thus there exist $\lambda_1,\dots, \lambda_n\in\R$ such that $Tv_k = \lambda_k v_k$ for $k = 1,\dots, n$.  Thus $\mat(T)$ is diagonal, and all entries along the diagonal are real.  Therefore $\mat(T)$ equals the conjugate transpose of $\mat(T)$.  By 7.10, this implies $\mat(T) = \mat(T^\ast)$, and we conclude $T = T^\ast$, so that $T$ is indeed self-adjoint.
\end{proof}

% Problem 7
\begin{problem}{7}
Suppose $V$ is a complex inner product space and $T\in\Hom(V)$ is a normal operator such that $T^9 = T^8$.  Prove that $T$ is self-adjoint and $T^2 = T$.   
\end{problem}
\begin{proof}
By the Complex Spectral Theorem, since $T$ is normal, $V$ has an orthonormal basis $v_1,\dots, v_n$ consisting of eigenvectors of $T$.  Let $\lambda_1,\dots, \lambda_n\in\C$ be the corresponding eigenvalues, so that
\begin{align*}
Tv_k = \lambda_k v_k
\end{align*}
for $k = 1,\dots, n$.  Repeatedly applying $T$ to both sides of the equation above $8$ times yields
\begin{align*}
T^9v_k = (\lambda_k)^9 v_k\quad\text{and}\quad T^8v_k = (\lambda_k)^8 v_k.
\end{align*}
Since $T^9 = T^8$, we conclude $(\lambda_k)^9 = (\lambda_k)^8$ and thus $\lambda_k\in\{0, 1\}$.  In particular, all eigenvalues of $T$ are real, hence by Problem 6 we have that $T$ is self-adjoint.\\
\indent To see that $T^2 = T$, notice 
\begin{align*}
T^2v_k &= (\lambda_k)^2v_k\\
&= \lambda_k v_k\\
&=Tv_k,
\end{align*}
where the second equality follows from the fact that $\lambda_k\in\{0,1\}$, and the proof is complete.
\end{proof}

% Problem 9
\begin{problem}{9}
Suppose $V$ is a complex inner product space.  Prove that every normal operator on $V$ has a square root.  (An operator $S\in\Hom(V)$ is called a \textbf{\textit{square root}} of $T\in\Hom(V)$ if $S^2 = T$.)
\end{problem}
\begin{proof}
Suppose $T\in\Hom(V)$ is normal.  By the Complex Spectral Theorem, $V$ has an orthonormal basis $v_1,\dots, v_n$ consisting of eigenvectors of $T$.  Let $\lambda_1,\dots, \lambda_n\in\C$ be the corresponding eigenvalues, so that
\begin{align*}
Tv_k = \lambda_k v_k
\end{align*}
for $k = 1,\dots, n$.  Define $S\in\Hom(V)$ by its action on this basis:
\begin{align*}
Sv_k = \sqrt{\lambda_k}v_k,
\end{align*}
choosing the complex square root $\sqrt{\lambda_k}$ by some definite rule.  Let $v\in V$.  Then there exist $\alpha_1,\dots, \alpha_n\in\C$ such that $v = \alpha_1v_1 + \dots + \alpha_nv_n$.  It follows
\begin{align*}
S^2 v &= S^2( \alpha_1v_1 + \dots + \alpha_nv_n)\\
&= S\left(\alpha_1\sqrt{\lambda_1}v_1 + \dots + \alpha_n\sqrt{\lambda_n}v_n\right)\\
&= \alpha_1\lambda_1v_1 + \dots + \alpha_n\lambda_n v_n\\
&= \alpha_1Tv_1 + \dots + \alpha_nTv_n\\
&= T( \alpha_1v_1 + \dots + \alpha_nv_n)\\
&= Tv.
\end{align*}
Thus $S^2 = T$, and indeed $T$ has a square root, as was to be shown.
\end{proof}

\begin{problem}{11}
Prove or give a counterexample: every self-adjoint operator on $V$ has a cube root.  (An operator $T\in\Hom(V)$ is called a \textbf{\textit{cube root}} of $T\in\Hom(V)$ if $S^3 = T$.)
\end{problem}
\begin{proof}
Suppose $T\in\Hom(V)$ is self-adjoint.  Regardless of whether $\F=\R$ or $\F =\R$, both Spectral Theorems imply that $V$ has an orthonormal basis $v_1,\dots, v_n$ consisting of eigenvectors of $T$.  By 7.13, all eigenvalues of $T$ are real.  So let $\lambda_1,\dots, \lambda_n\in\R$ be the eigenvalues corresponding to $v_1,\dots, v_n$, so that
\begin{align*}
Tv_k = \lambda_k v_k
\end{align*}
for $k = 1,\dots, n$.  Define $S\in\Hom(V)$ by its action on this basis:
\begin{align*}
Sv_k = (\lambda_k)^\frac{1}{3}v_k,
\end{align*}
Let $v\in V$.  Then there exist $\alpha_1,\dots, \alpha_n\in\C$ such that $v = \alpha_1v_1 + \dots + \alpha_nv_n$.  It follows
\begin{align*}
S^3 v &= S^3( \alpha_1v_1 + \dots + \alpha_nv_n)\\
&= S^2\left(\alpha_1(\lambda_1)^\frac{1}{3}v_1 + \dots + \alpha_n(\lambda_n)^\frac{1}{3}v_n\right)\\
&= S\left(\alpha_1(\lambda_1)^\frac{2}{3}v_1 + \dots + \alpha_n(\lambda_n)^\frac{2}{3} v_n\right)\\
&= \alpha_1\lambda_1v_1 + \dots + \alpha_n\lambda_n v_n\\
&= \alpha_1Tv_1 + \dots + \alpha_nTv_n\\
&= T( \alpha_1v_1 + \dots + \alpha_nv_n)\\
&= Tv.
\end{align*}
Thus $S^3 = T$, and indeed $T$ has a cube root.  Thus, all self-adjoint operators on a finite-dimensional inner product space have a cube root.
\end{proof}

% Problem 13
\begin{problem}{13}
Give an alternative proof of the Complex Spectral Theorem that avoids Schur's Theorem and instead follows the pattern of the proof of the Real Spectral Theorem.
\end{problem}
\begin{proof}
Suppose (c) holds, so that $T$ has a diagonal matrix with respect to some orthonormal basis of $V$.  The matrix of $T^\ast$ (with respect to the same basis) is obtained by taking the conjugate transpose of the matrix of $T$; hence $T^\ast$ also has a diagonal matrix.  Any two diagonal matrices commute; thus $T$ commutes with $T^\ast$, which means that $T$ is normal.  That is, (a) holds.\\
\indent We will prove that (a) implies (b) by induction on $\dim V$.  For our base case, suppose $\dim V= 1$.  Since 5.21 guarantees the existence of an eigenvector of $T$, clearly (b) is true in this case.  Next assume that $\dim V > 1$ and that (a) implies (b) for all complex inner product spaces of smaller dimension.\\
\indent Suppose (a) holds, so that $T$ is normal.  Let $u$ be an eigenvector of $T$ with $\norm{u} = 1$, and set $U = \Span(u)$.  Clearly $U$ is invariant under $T$.  By Problem 3 of 7A, this implies that $U^\perp$ is invariant under $T^\ast$ as well.  But of course $T^\ast$ is also normal, and since $\dim U^\perp = \dim V - 1$, our inductive hypothesis implies that there exists an orthonormal basis of $U^\perp$ consisting of eigenvectors of $T\mid_{U^\perp}$.  Adjoining $u$ to this basis gives an orthonormal basis of $V$ consisting of eigenvectors of $T$, completing the proof that (a) implies (b).\\
\indent We have proved that (c) implies (a) and that (a) implies (b).  Clearly (b) implies (c), and the proof is complete.
\end{proof}

% Problem 15
\begin{problem}{15}
Find the value of $x$ such that the matrix
\begin{align*}
\begin{bmatrix}
1 & 1 & 0 \\
0 & 1 & 1 \\
1 & 0 & x 
\end{bmatrix}
\end{align*}
is normal.
\end{problem}
\begin{proof}
Let $M$ be the above matrix.  We wish to find $x\in \F$ such that $MM^\ast = M^\ast M$.  Notice
\begin{align*}
MM^\ast &= \begin{bmatrix}
1 & 1 & 0 \\
0 & 1 & 1 \\
1 & 0 & x 
\end{bmatrix}
\begin{bmatrix}
1 & 0 & 1 \\
1 & 1 & 0 \\
0 & 1 & x 
\end{bmatrix}\\
&= 
\begin{bmatrix}
2 & 1 & 1 \\
1 & 2 & x \\
1 & x & 1 + x^2 
\end{bmatrix}
\end{align*}
and
\begin{align*}
M^\ast M &= 
\begin{bmatrix}
1 & 0 & 1 \\
1 & 1 & 0 \\
0 & 1 & x 
\end{bmatrix}
\begin{bmatrix}
1 & 1 & 0 \\
0 & 1 & 1 \\
1 & 0 & x 
\end{bmatrix}\\
&= 
\begin{bmatrix}
2 & 1 & x \\
1 & 2 & 1 \\
x & 1 & 1 + x^2 
\end{bmatrix}.
\end{align*}
Thus it must be that $x = 1$.
\end{proof}
\end{document}